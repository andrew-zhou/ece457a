\documentclass[conference]{IEEEtran}
\usepackage{cite}
\usepackage{amsmath,amssymb,amsfonts}
\usepackage{algorithmic}
\usepackage{graphicx}
\usepackage{textcomp}
\usepackage[english]{babel}
\usepackage{blindtext}
\def\BibTeX{{\rm B\kern-.05em{\sc i\kern-.025em b}\kern-.08em
    T\kern-.1667em\lower.7ex\hbox{E}\kern-.125emX}}
\begin{document}

\title{Comparison between Genetic Algorithms and Ant Colony Optimization for Multi-Agent Path Planning in 3D}

\author{
\IEEEauthorblockN{Cheng Dong}
\IEEEauthorblockA{\textit{fkmansou@uwaterloo.ca}}
\and
\IEEEauthorblockN{Zhaotian Fang}
\IEEEauthorblockA{\textit{z23fang@edu.uwaterloo.ca}}
\and
\IEEEauthorblockN{Zu Qi Li}
\IEEEauthorblockA{\textit{zq6li@edu.uwaterloo.ca}}
\and
\IEEEauthorblockN{Di Sen Lu}
\IEEEauthorblockA{\textit{fkmansou@uwaterloo.ca}}
\and
\IEEEauthorblockN{YuYang Si}
\IEEEauthorblockA{\textit{y6si@uwaterloo.ca}}
\and
\IEEEauthorblockN{Andrew Zhou}
\IEEEauthorblockA{\textit{a25zhou@edu.uwaterloo.ca}}
}

\maketitle

\begin{abstract}
% TODO: Zuqi
\blindtext
\end{abstract}

\begin{IEEEkeywords}
% Source: https://www.ieee.org/documents/taxonomy_v101.pdf
ant colony optimization, genetic algorithms, path planning, network theory (graphs), cost function, parallel programming
\end{IEEEkeywords}

\section{Introduction}
% TODO: David
\blindtext

\section{Literature Review}
Previous research in the field of three dimensional path planning has focused on the single-agent problem and has traditionally looked at genetic algorithm (GA) and particle swarm optimization (PSO) approaches. Studies in this field have motivated the techniques utilised in this paper both through the algorithms for single-drone path finding as well as the reduction of a real three-dimensional space into a graph of waypoints.

A key optimization conducted in this paper is the reduction of real three-dimensional spaces into adjacency lists of waypoints. Doing this minimizes the search space drastically, enabling the path planning techniques to be more effective on a wider range of potential spaces. This reduction is done by grouping neighbouring points into waypoints and then calculating an average real-world cost to move between neighbouring waypoints. Studies on real-time UAV path planning suggest that the cost for a drone to move between points is determined by distance, altitude, power consumption, fuel usage, ground collisions, and whether or not the drone must traverse through danger zones \cite{b1}. To translate this from points to waypoints, this paper treats the position of the waypoint as the center point within the space that it occupies. Then the resulting cost function is:
$$F_{cost} = C_{length} + C_{altitude} + C_{danger zones}$$
$$ + C_{power} + C_{collision} + C_{fuel}$$

Prior studies have thus far been inconclusive in determining whether GA techniques are superior to PSO techniques for the single-drone problem \cite{b2} or vice-versa \cite{b3}. Additionally, the addition of multiple simultaneous drones adds a major consideration which these techniques do not take into account. This lack of conclusiveness prevented this paper's techniques from being fully rooted in the traditional methods discussed in the literature. Nevertheless, those approaches are taken into account.

The multi-drone path planning techniques discussed in this paper break down the problem into a series of single-drone path planning subproblems. These subproblems are solved using a combination of techniques similar to the literature (GA-based) as well as more novel approaches through ant colony optimization (ACO) methods. The methods for reducing the multi-drone problem are themselves unique and not considered in prior literature.

\section{Problem Formulation and Modeling}
Based on the papers we reviewed, we decided to reduce the problem from a three dimensional space coordinate system to a graph of waypoints. By reducing the problem this way, we ended up with a basic graph traversal problem which reduces the search space of the problem drastically as it is now essentially a two dimensional problem. The graph is represented as an adjacency list to have a compact way of storing the graph in memory at the cost of edge look-up time. Each waypoint in the graph represents a group of neighbouring points in three dimensional space whereas the cost to move from one waypoint to another is calculated by taking the average of the real world cost between the waypoints. To reduce the search space into two dimensions, the cost function accounts for the altitude changes as well as other important factors like distance, danger zones and power consumption. Therefore, our cost function can be expressed as:
$$F_{cost} = C_{distance} + C_{altitude} + C_{danger zones} + C_{power}$$
The cost for distance, power and altitude are fairly self explanatory as they are calculated by measuring the power consumption, distance and altitude that the drone has to travel between points A and B.

Danger zones of areas that the drone should try to avoid such as high traffic areas, areas that are close to buildings or infrastructure. The drone will try to avoid danger zones if possible but  The danger zone calculation is as follows: 

$$ C_{danger zones} = \frac{L_{inside}}{\sum_{i=1}^{n} d_{i}} $$

The calculation consists of $L_{inside}$, which is the portion of path that is inside a danger zone and the summation of $d_{i}$, which is the diameter of each danger zone. This way, the path is penalized more for passing through when there are a few danger zones as opposed to many danger zones \cite{b1}. 

\section{Proposed Solution}
% TODO: Jason
\blindtext

\section{Performance Evaluation}
% TODO: Tony
\blindtext

\section{Conclusions \& Recommendations}
% TODO: Zuqi
\blindtext

\section*{References}

\begin{thebibliography}{00}
\bibitem{b1} Roberge, Vincent, Mohammed Tarbouchi, and Gilles Labont. "Comparison of parallel genetic algorithm and particle swarm optimization for real-time UAV path planning." IEEE Transactions on Industrial Informatics 9.1 (2013): 132-141
\bibitem{b2} Tu, Jianping, and Simon X. Yang. "Genetic algorithm based path planning for a mobile robot." Robotics and Automation, 2003. Proceedings. ICRA'03. IEEE International Conference on. Vol. 1. IEEE, 2003
\bibitem{b3} Masehian, Ellips, and DavoudSedighizadeh. "A multi-objective PSO-based algorithm for robot path planning." Industrial Technology (ICIT), 2010 IEEE International Conference on. IEEE, 2010
\end{thebibliography}

\end{document}
